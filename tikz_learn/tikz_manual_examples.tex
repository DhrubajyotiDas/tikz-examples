\documentclass[letterpaper, headinclude, footinclude = true]{article}

\PassOptionsToPackage{noopticals, mathlf, roundv}{MinionPro}
\usepackage[minionprospacing, nochapters, beramono, listings, minionpro]{classicthesis}
%\DeclareOption{eulermath}{\setboolean{@eulermath}{false}} 

\usepackage{amsmath}
\usepackage[margin = 1in]{geometry}
\usepackage{tikz}
\usepackage{listings}



\begin{document}

\lstset{language=[LaTeX]Tex,%C++,
    keywordstyle=\color{RoyalBlue},%\bfseries,
    basicstyle=\small\ttfamily,
    %identifierstyle=\color{NavyBlue},
    commentstyle=\color{Green}\ttfamily,
    stringstyle=\rmfamily,
    numbers=none,%left,%
    numberstyle=\scriptsize,%\tiny
    stepnumber=5,
    numbersep=8pt,
    showstringspaces=false,
    breaklines=true,
    frameround=ftff,
    %frame=single,
    belowcaptionskip=.75\baselineskip
    %frame=L
} 

\usetikzlibrary{intersections, angles, quotes}


\section{A Trigonometry Example} % (fold)
\label{sec:a_trigonometry_example}

% section a_trigonometry_example (end)
\begin{tikzpicture}[scale = 3, line cap = round, >=stealth,
% Styles
axes/.style = ,
important line/.style = {very thick},
information text/.style = {rounded corners, fill = red!10, inner sep = 1ex}]

% Colors
\colorlet{anglecolor}{green!50!blue}
\colorlet{sincolor}{red}
\colorlet{tancolor}{orange!90!black}
\colorlet{coscolor}{blue}

% The graphic
\draw [help lines, step = 0.5cm] (-1.4, -1.4) grid (1.4, 1.4);

\draw (0,0) circle [radius = 1 cm];

\begin{scope}[axes]
    \draw [->] (-1.5, 0) -- (1.5, 0) node [right] {$x$} coordinate (x axis);
    \draw [->] (0, -1.5) -- (0, 1.5) node [above] {$y$} coordinate (y axis);

    \foreach \x / \xmath in {-1, -0.5 / -\frac{1}{2}, 1}
        \draw (\x, 1pt) -- (\x, -1pt) node [below, fill = white] {$\xmath$};
    \foreach \y / \ymath in {-1, -0.5/ -\frac{1}{2} , 0.5 / \frac{1}{2}, 1}    
        \draw (1pt, \y) -- (-1pt, \y) node [left, fill = white] {$\ymath$};
\end{scope}

\filldraw[fill = green!20, draw = anglecolor] (0,0) -- (0.3, 0)
    arc [start angle = 0, end angle = 30, radius = 3 mm] -- cycle;
\draw (15:2mm) node [anglecolor] {$\alpha$};

\draw[important line, sincolor]
    (30:1)	-- node[left, fill = white] {$\sin\alpha$} (30:1 |- x axis);

\draw[important line, coscolor]
    (0,0)  -- node[below, fill = white] {$\cos\alpha$} (30:1 |- x axis);    

\path [name path = upward line] (1,0) -- (1,1);
\path [name path = slanted line] (0,0) -- (30:2);
\draw [important line, tancolor,
        name intersections = {of = upward line and slanted line, by = t}] 
        (1,0) -- node [right, fill = white] {$\tan\alpha \color{black} = \dfrac{\color{red}\sin\alpha}{\color{blue}\cos\alpha}$} (t);

\draw (0,0) -- (t);

% Informational text
\draw[xshift = 1.85 cm]
    node [right, text width = 6 cm, information text]
    {
        The {\color{anglecolor} angle $\alpha$} is $30^\circ$ in the example ($\pi/6$ in radians). The {\color{sincolor} sine of $\alpha$}, which is the height of the red line, is \[{\color{sincolor} \sin \alpha} = \frac{1}{2}.\]
        By the Theorem of Pythagoras we have {\color{coscolor} $\cos^2\alpha$} + {\color{sincolor} $\sin^2 \alpha$} = 1. Thus the length of the blue line, which is the {\color{coscolor} cosine of $\alpha$}, must be \[{\color{coscolor} \cos \alpha} = \sqrt{1-\frac{1}{4}} = \frac{1}{2}\sqrt{3}.\]
        This shows that {\color{tancolor} $\tan \alpha$}, which is the height of the orange line, is \[{\color{tancolor} \tan \alpha} = \frac{\color{sincolor}\sin \alpha}{\color{coscolor}\cos\alpha} = \frac{1}{\sqrt{3}}.\]
    };




\end{tikzpicture}


g $g$

$v$

$\nu$

$w$

$\upsilon$



\end{document}